\section{Introduction}
\label{sect:introduction}

\subsection{Prerequisites}
\label{ssect:prerequisites}
This tutorial is aimed towards code developers and researchers, thus familiarity with Linux terminal is assumed. Unfortunately, this is not aimed to be a thorough tutorial on python, Quantum GIS or PyQt alone. It will show the reader how to bring these ``elements'' together. I appreciate that some of the readers may be beginners on any of these, and I have tried to explain many basic concepts. However, the beginner will find that I usually encourage them to research and understand the problem, as well as just giving the code. In addition, there are many good tutorials on python QGIS and PyQt available on-line, so I would be re-inventing the wheel. As such, the beginner is encouraged to look into the following:
\begin{enumerate}
  \item For begginners to Qt / PyQt:
  \begin{enumerate}
    \item Qt was created by Digea as a C++ library. More information can be found at \url{http://qt.digia.com/}.
    \item PyQt is a set of python bindings to the Qt library. It was developed by River Bank Computing, \url{http://www.riverbankcomputing.com/software/pyqt/intro}. PyQt is not the only python binding of Qt, there are others. However, since QGIS uses PyQt, there is no reason to examine them here.
    \item \url{http://zetcode.com/tutorials/pyqt4/} contains an excellent PyQt tutorial.
    \item An exellent reference to the PyQt classes can be found at \url{http://pyqt.sourceforge.net/Docs/PyQt4/classes.html}.
  \end{enumerate}
  \item List to be finished off...
  \begin{enumerate}
    \item \url{http://docs.python.org/2/howto/unicode.html}
  \end{enumerate}
\end{enumerate}

\subsection{What are QGIS plugins?}
\par
A framework to execute arbitrary, user-defined code, while providing access to QGIS internals. Mention repositories.\\
QGIS-plugins are implemented as python packages : Quick introduction of the structure of a python package, give references to further material.\\
QGIS-plugins also use a graphical interface framework: PyQt. Quick introduction.\\
Sketch the various components of a QGIS-plugin: How QGIS, python-packages and PyQt are brought together.

This is not a QGIS installation guide, nonetheless Ubuntu users should be able to add the necessary repository via:
\begin{lstlisting}
sudo apt-get install python-software-properties
sudo add-apt-repository ppa:ubuntugis/ppa
\end{lstlisting}
